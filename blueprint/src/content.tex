% In this file you should put the actual content of the blueprint.
% It will be used both by the web and the print version.
% It should *not* include the \begin{document}
%
% If you want to split the blueprint content into several files then
% the current file can be a simple sequence of \input. Otherwise It
% can start with a \section or \chapter for instance.

\section{section1}
\begin{definition}
  \label{def:E8Lattice}
  \lean{Project.IntegralLattice.E8.Basic.E8Lattice}
  \leanok
  \uses{def:IntegralLattice}
  $E_8$格子とは,integralLatticeであって,even unimodularかつランクが$8$であるもののことである.
\end{definition}

\begin{theorem}
  \label{thm:unique}
  \lean{Project.IntegralLattice.E8.Basic.E8Lattice.unique}
  \leanok
  2つの$E_8$格子$\Lambda_1,\ \Lambda_2$は同型である.
\end{theorem}

\begin{proof}
  sorry.
\end{proof}

\begin{definition}
  \label{def:M0-M7}
  \lean{Project.IntegralLattice.E8.Basic.E8Lattice.M0}
  \leanok
  $E_8$のCartan行列を$M_0$,それを1行ずつ行基本変形していき(その過程の行列を$M_1, M_2, \ldots, M_6$とする)上三角にしたものを$M_7$とする:
  \begin{gather}
    M_0 :=
    \begin{pmatrix}
      2 & 0 & -1 & 0 & 0 & 0 & 0 & 0 \\
      0 & 2 & 0 & -1 & 0 & 0 & 0 & 0 \\
      -1 & 0 & 2 & -1 & 0 & 0 & 0 & 0 \\
      0 & -1 & -1 & 2 & -1 & 0 & 0 & 0 \\
      0 & 0 & 0 & -1 & 2 & -1 & 0 & 0 \\
      0 & 0 & 0 & 0 & -1 & 2 & -1 & 0 \\
      0 & 0 & 0 & 0 & 0 & -1 & 2 & -1 \\
      0 & 0 & 0 & 0 & 0 & 0 & -1 & 2
    \end{pmatrix},\\
    M_7 :=
    \begin{pmatrix}
      2 & 0 & -1 & 0 & 0 & 0 & 0 & 0 \\
      0 & 2 & 0 & -1 & 0 & 0 & 0 & 0 \\
      0 & 0 & 3/2 & -1 & 0 & 0 & 0 & 0 \\
      0 & 0 & 0 & 5/6 & -1 & 0 & 0 & 0 \\
      0 & 0 & 0 & 0 & 4/5 & -1 & 0 & 0 \\
      0 & 0 & 0 & 0 & 0 & 3/4 & -1 & 0 \\
      0 & 0 & 0 & 0 & 0 & 0 & 2/3 & -1 \\
      0 & 0 & 0 & 0 & 0 & 0 & 0 & 1/2
    \end{pmatrix}.
  \end{gather}
\end{definition}

\begin{lemma}
  \label{lem:M7_upperTrianglar}
  \lean{Project.IntegralLattice.E8.Basic.E8Lattice.M7_upperTrianglar}
  \leanok
  $M_7$は上三角である.
\end{lemma}

\begin{proof}
  \leanok
  \uses{def:M0-M7}
  略.
\end{proof}

\begin{lemma}
  \label{lem:M7_det}
  \lean{Project.IntegralLattice.E8.Basic.E8Lattice.M7_det}
  \leanok
  $\det M_7 = 1$である.
\end{lemma}

\begin{proof}
  \leanok
  \uses{lem:M7_upperTrianglar, def:M0-M7}
  補題~\ref{lem:M7_upperTrianglar}より,$M_7$の行列式は対角成分たちの積であるから
  \begin{equation}
    \det M_7
    = 2 \cdot 2 \cdot (3 / 2) \cdot (5 / 6) \cdot (4 / 5) \cdot (3 / 4) \cdot (2 / 3) \cdot (1 / 2)\\
    = 1.
  \end{equation}
\end{proof}

\begin{theorem}
  \label{thm:E8_det}
  \lean{Project.IntegralLattice.E8.Basic.E8Lattice.E8_det}
  \leanok
  $E_8$のCartan行列の行列式は$1$である.
\end{theorem}

\begin{proof}
  \uses{def:M0-M7, lem:M7_det}
  補題~\ref{lem:M7_det}より
  \begin{equation}
    (\textrm{求める行列式})
    = \det M_0
    = \det M_1
    = \cdots
    = \det M_7
    = 1.
  \end{equation}
\end{proof}

\begin{definition}
  \label{def:B}
  \lean{Project.IntegralLattice.E8.Basic.E8Lattice.B}
  \leanok
  $B$を$E_8$のCartan行列$C (= M_0) \in \M_8(\Z)$から定まる双線型形式とする:
  \begin{equation}
    B(x, y) := \transpose{x} C y = \inner{x}{Cy} \qquad (\forall x, y \in \Z^8).
  \end{equation}
\end{definition}

\begin{lemma}
  \label{lem:inner_self_calc}
  \lean{Project.IntegralLattice.E8.Basic.E8Lattice.inner_self_calc}
  \leanok
  任意の$x \in \Z^8$に対し,次が成り立つ:
  \begin{equation}
    \begin{split}
      B(x, x) ={}& 2 (x_0^2 + x_1^2 + x_2^2 + x_3^2 + x_4^2 + x_5^2 + x_6^2 + x_7^2\\
      &- (x_0 x_2 + x_1 x_3 + x_2 x_3 + x_3 x_4 + x_4 x_5 + x_5 x_6 + x_6 x_7))
    \end{split}
  \end{equation}
\end{lemma}

\begin{proof}
  \leanok
  \uses{def:B}
  内積の形にして,あとは具体的に計算:
  \begin{equation}
    B(x, x)
    = \inner{x}{Cx}
    = (\textrm{右辺}).
  \end{equation}
\end{proof}

\begin{lemma}
  \label{lem:inner_self_comp_sq}
  \lean{Project.IntegralLattice.E8.Basic.E8Lattice.inner_self_comp_sq}
  \leanok
  任意の$x \in \Z^8$に対し,平方完成すると次のようになる:
  \begin{equation}
    \begin{split}
      B(x, x) ={}& \left( \sqrt{2} \, x_0 - \sqrt{\frac{1}{2}} \, x_2 \right)^2 + \left( \sqrt{2} \, x_1 - \sqrt{\frac{1}{2}} \, x_3 \right)^2 + \left( \sqrt{\frac{3}{2}} \, x_2 - \sqrt{\frac{2}{3}} \, x_3 \right)^2 \\
      & + \left( \sqrt{\frac{5}{6}} \, x_3 - \sqrt{\frac{6}{5}} \, x_4 \right)^2 + \left( \sqrt{\frac{4}{5}} \, x_4 - \sqrt{\frac{5}{4}} \, x_5 \right)^2 + \left( \sqrt{\frac{3}{4}} \, x_5 - \sqrt{\frac{4}{3}} \, x_6 \right)^2 \\
      & + \left( \sqrt{\frac{2}{3}} \, x_6 - \sqrt{\frac{3}{2}} \, x_7 \right)^2 + \frac{1}{2} \, x_7^2
    \end{split}
  \end{equation}
\end{lemma}

\begin{proof}
  \leanok
  \uses{lem:inner_self_calc}
  左辺に補題~\ref{lem:inner_self_calc}を代入して計算すれば得られる.
\end{proof}

\begin{theorem}
  \label{thm:add_inner}
  \lean{Project.IntegralLattice.E8.Basic.E8Lattice.add_inner}
  \leanok
  $\forall x, y, z \in \Z^8,\ B(x+y, z) = B(x, z) + B(y, z)$.
\end{theorem}

\begin{proof}
  \leanok
  \uses{def:B}
  計算するだけ.
\end{proof}

\begin{theorem}
  \label{thm:inner_sym}
  \lean{Project.IntegralLattice.E8.Basic.E8Lattice.inner_sym}
  \leanok
  $\forall x, y \in \Z^8,\ B(x, y) = B(y, x)$.
\end{theorem}

\begin{proof}
  \leanok
  \uses{def:B}
  計算するだけ.
\end{proof}

\begin{theorem}
  \label{thm:inner_self}
  \lean{Project.IntegralLattice.E8.Basic.E8Lattice.inner_self}
  \leanok
  $\forall x \in \Z^8,\ B(x, x) \ge 0$.
\end{theorem}

\begin{proof}
  \leanok
  \uses{lem:inner_self_comp_sq}
  補題~\ref{lem:inner_self_comp_sq}より,$B(x, x)$は平方の和で表せるから成り立つ.
\end{proof}

\begin{theorem}
  \label{thm:inner_self_eq_zero}
  \lean{Project.IntegralLattice.E8.Basic.E8Lattice.inner_self_eq_zero}
  \leanok
  $\forall x \in \Z^8,\ B(x, x) = 0 \implies x = 0$.
\end{theorem}

\begin{proof}
  \leanok
  \uses{lem:inner_self_comp_sq}
  補題~\ref{lem:inner_self_comp_sq}より,$B(x, x)$は平方の和で表せ,$= 0$とすると各項が$0$である.
  よって,最後の項に注目すると,$x_7 = 0$である.
  したがって,最後から2番目の項に注目すると,$x_6 = 0$である.
  これを繰り返すと,$x_0 = \cdots = x_7 = 0$を得る.
\end{proof}

\begin{theorem}
  \label{thm:even}
  \lean{Project.IntegralLattice.E8.Basic.E8Lattice.even}
  \leanok
  $\forall x \in \Z^8,\ 2 \mid \inner{x}{x}_{\Z}$.
\end{theorem}

\begin{proof}
  \leanok
  \uses{lem:inner_self_calc}
  補題~\ref{lem:inner_self_calc}より従う.
\end{proof}

\begin{theorem}
  \label{thm:unimodular}
  \lean{Project.IntegralLattice.E8.Basic.E8Lattice.unimodular}
  \leanok
  $E_8$格子はunimodularである.
\end{theorem}

\begin{proof}
  sorry.
\end{proof}

\begin{theorem}
  \label{thm:exists_E8}
  \lean{Project.IntegralLattice.E8.Basic.E8Lattice.exists_E8}
  \leanok
  $E_8$格子は存在する.
\end{theorem}

\begin{proof}
  sorry.
\end{proof}

\begin{theorem}
  $E_8$格子$\Lambda$に対し,$\forall n \in \N,\ \#\set{x \in \Lambda}{B(x, x) = n} < \infty$.
\end{theorem}

\begin{proof}
  sorry.
\end{proof}

\begin{lemma}
  \label{lem:card_norm_2}
  \lean{Project.IntegralLattice.E8.Basic.E8Lattice.card_norm_2}
  \leanok
  $E_8$格子$\Lambda$に対し,$\#\set{x \in \Lambda}{B(x, x) = 2} = 240$.
\end{lemma}

\begin{proof}
  sorry.
\end{proof}
