\documentclass[unicode,11pt]{ltjsarticle}
\usepackage{luatexja}
\usepackage{amsmath,amssymb,amsthm,mathtools,mathcomp,empheq,bm,enumitem,url,tikz,graphicx,color,tcolorbox,listings,newverbs,hhline,multirow}
\usepackage[no-math]{fontspec}

\usetikzlibrary{intersections,calc,angles,quotes}

% 和文のフォントをIPAexに
\usepackage[no-math,deluxe,ipaex]{luatexja-preset}

% ハイパーリンクの設定
\usepackage[pdfencoding=auto]{hyperref}

% ttfamily (タイプライタ)のフォントを変更
\setmonofont{Menlo}

\theoremstyle{definition}
\newtheorem{defi}{定義}[section]
\newtheorem{thm}[defi]{定理}
\newtheorem{prop}[defi]{命題}
\newtheorem{lem}[defi]{補題}
\newtheorem{cor}[defi]{系}
\newtheorem{rem}[defi]{注意}
\newtheorem{exa}[defi]{例}
\newtheorem{exe}[defi]{演習}
\newtheorem{fac}[defi]{事実}
\newtheorem{que}{問題}
\newtheorem{ans}{解答}
\newtheorem*{defi*}{定義}
\newtheorem*{thm*}{定理}
\newtheorem*{prop*}{命題}
\newtheorem*{lem*}{補題}
\newtheorem*{cor*}{系}
\newtheorem*{rem*}{注意}
\newtheorem*{exa*}{例}
\newtheorem*{exe*}{演習}
\newtheorem*{fac*}{事実}
\newtheorem*{hosoku*}{補足}

\newcommand{\st}{\ \text{s.t.}\ }
\newcommand{\map}[2]{#1 \rightarrow #2}
\newcommand{\set}[2]{\left\lbrace #1 \mathrel{} \middle| \mathrel{} #2 \right\rbrace}
\newcommand{\ddv}[2]{\dfrac{d #1}{d #2}}
\newcommand{\dpdv}[2]{\dfrac{\partial #1}{\partial #2}}
\newcommand{\dsum}{\displaystyle \sum}
\newcommand{\dlim}{\displaystyle \lim}
\newcommand{\dint}{\displaystyle \int}
\newcommand{\abs}[1]{\left| #1 \right|}
\newcommand{\norm}[1]{\left\| #1 \right\|}
\newcommand{\inner}[2]{\left\langle #1, #2 \right\rangle}
\newcommand{\transpose}[1]{{}^t\!#1}
\newcommand{\C}{\mathbb{C}}
\newcommand{\R}{\mathbb{R}}
\newcommand{\Q}{\mathbb{Q}}
\newcommand{\Z}{\mathbb{Z}}
\newcommand{\N}{\mathbb{N}}
\newcommand{\V}{\mathbb{V}}
\newcommand{\I}{\mathbb{I}}
\newcommand{\F}{\mathbb{F}}
\newcommand{\M}{\operatorname{M}}
\newcommand{\GL}{\operatorname{GL}}
\newcommand{\SL}{\operatorname{SL}}
\newcommand{\tr}{\operatorname{tr}}
\newcommand{\Ker}{\operatorname{Ker}}
\newcommand{\Aut}{\operatorname{Aut}}
\newcommand{\End}{\operatorname{End}}
\newcommand{\ann}[1]{\operatorname{Ann}\, ({#1})}
\newcommand{\Co}{\operatorname{Co}}
\DeclareMathOperator{\id}{id}
\DeclareMathOperator*{\esssup}{ess\,sup}

% figureのキャプションをsectionに関連づける
\counterwithin{figure}{section}

% 式番号をsectionと関連づける
\numberwithin{equation}{section}

% 数式番号を参照されたもののみ表示
\mathtoolsset{showonlyrefs=true}

% 証明環境の見出しを日本語に
\def\proofname{\textbf{証明}}

% 証明環境の再定義
\makeatletter
\renewenvironment{proof}[1][\proofname]{\par
\pushQED{\qed}%
\normalfont \topsep6\p@\@plus6\p@\relax
\trivlist
\item\relax
{\bfseries
#1\@addpunct{\textbf{.}}}\hspace\labelsep\ignorespaces
}{
\popQED\endtrivlist\@endpefalse
}
\makeatother

% 証明環境の最後を黒四角に
\renewcommand{\qedsymbol}{$\blacksquare$}

% 脚注番号をアラビア数字に
\renewcommand{\thefootnote}{\arabic{footnote}}

% 大きい行列を扱うため
\setcounter{MaxMatrixCols}{24}

\title{レポート(有限鏡映群)}
\author{苗代 昇(学生番号: 20243029)}
\date{}

\begin{document}
\maketitle

ここでは内積を$\inner{\cdot}{\cdot}$と書く.

\begin{ans}
\begin{proof}
  $\lambda, \mu \in V$とし,$\lambda = c_\lambda \alpha + \beta_\lambda,\ \mu = c_\mu \alpha + \beta_\mu$と書く($c_\lambda, c_\mu \in \R,\ \beta_\lambda, \beta_\mu \in H_\alpha$).
  このとき,
  \begin{align}
    \inner{s_\alpha(\lambda)}{s_\alpha(\mu)}
    &= \inner{s_\alpha(c_\lambda \alpha + \beta_\lambda)}{s_\alpha(c_\mu \alpha + \beta_\mu)}\\
    &= \inner{-c_\lambda \alpha + \beta_\lambda}{-c_\mu \alpha + \beta_\mu}\\
    &= c_\lambda c_\mu \inner{\alpha}{\alpha} - c_\lambda \inner{\alpha}{\beta_\mu} - c_\mu \inner{\beta_\lambda}{\alpha} + \inner{\beta_\lambda}{\beta_\mu}\\
    &= c_\lambda c_\mu \inner{\alpha}{\alpha} + \inner{\beta_\lambda}{\beta_\mu}.
  \end{align}
  一方,
  \begin{align}
    \inner{\lambda}{\mu}
    &= \inner{c_\lambda \alpha + \beta_\lambda}{c_\mu \alpha + \beta_\mu}\\
    &= c_\lambda c_\mu \inner{\alpha}{\alpha} + c_\lambda \inner{\alpha}{\beta_\mu} + c_\mu \inner{\alpha}{\beta_\lambda} + \inner{\beta_\lambda}{\beta_\mu}\\
    &= c_\lambda c_\mu \inner{\alpha}{\alpha} + \inner{\beta_\lambda}{\beta_\mu}.
  \end{align}
  よって,$\inner{s_\alpha(\lambda)}{s_\alpha(\mu)} = \inner{\lambda}{\mu}$が成り立つから,$s_\alpha$は直交変換であることわかった.
\end{proof}
\end{ans}

\begin{ans}
  \begin{proof}
    (1)を確かめる:
    \begin{equation}
      f(\alpha)
      = \alpha - 2 \frac{\inner{\alpha}{\alpha}}{\inner{\alpha}{\alpha}} \alpha
      = \alpha - 2\alpha
      = -\alpha.
    \end{equation}
    (2)を確かめる.$\lambda \in H_\alpha$とすると,$\inner{\lambda}{\alpha} = 0$であるから,
    \begin{equation}
      f(\lambda)
      = \lambda - 2\frac{\inner{\lambda}{\alpha}}{\inner{\alpha}{\alpha}} \alpha
      = \lambda - (2 \cdot 0) \alpha
      = \lambda.
    \end{equation}
    以上より,$s_\alpha = f$である.
  \end{proof}
\end{ans}

\begin{ans}
  $\inner{\alpha}{\beta_1} = \inner{\alpha}{\beta_2} = 0$であるから
  \begin{equation}
    \begin{pmatrix}
      s_\alpha(\alpha) & s_\alpha(\beta_1) & s_\alpha(\beta_2)
    \end{pmatrix}
    = \begin{pmatrix}
      -\alpha & \beta_1 & \beta_2
    \end{pmatrix}
    = \begin{pmatrix}
      \alpha & \beta_1 & \beta_2
    \end{pmatrix}
    \begin{pmatrix}
      -1 & 0 & 0\\
      0 & 1 & 0\\
      0 & 0 & 1
    \end{pmatrix}
  \end{equation}
  である.
  よって,$\{\alpha, \beta_1, \beta_2\}$に関する$s_\alpha$の表現行列は$\begin{pmatrix}
    -1 & 0 & 0\\
    0 & 1 & 0\\
    0 & 0 & 1
  \end{pmatrix}$である.
\end{ans}

\end{document}
